%%%%%%%%%%%%%%%%%%%%%%%%%%%%%%%%%%%%%%%%%%%%%%%%%%%%%%%%%%%%%%%%%%%%
%           Auteur :  P.TRAN BA, E.BOUTTIER, J.SHI, E.ABIA         %
%         Création :  18/01/2012 17:05                             %
%%%%%%%%%%%%%%%%%%%%%%%%%%%%%%%%%%%%%%%%%%%%%%%%%%%%%%%%%%%%%%%%%%%%

\documentclass[a4paper,11pt]{article}

%\usepackage[utf8]{inputenc}
%\usepackage[T1]{fontenc}
%\usepackage{xunicode}
\usepackage{fontspec}
\defaultfontfeatures{Mapping=tex-text,Scale=MatchLowercase}
\usepackage{a4wide}
\usepackage{verbatim}
\usepackage{polyglossia}
\setdefaultlanguage{french}
%~ \usepackage{listings}
%\usepackage[french]{babel}
%~ \usepackage{url}
%~ \usepackage{times}
\usepackage{minted}
\usepackage{amssymb}
\usepackage{graphicx}
\input{graphviz}

\usepackage{geometry}
\geometry{hmargin=2.5cm,vmargin=1.5cm}

\title{Projet C\\Simulation d'un commutateur de niveau 3\\Rapport}
\author{Philippe \textsc{Tran Ba}, Élie \textsc{Bouttier}, Jiajun \textsc{Shi}, Émilie \textsc{Abia}\\Groupe 15}
\date\today

\begin{document}

\maketitle

\begin{abstract}

Ce rapport porte sur un projet de programmation d'un commutateur de niveau 3 en langage C. Il y est abordé le fonctionnement global du programme, suivi des choix techniques effectués afin de palier aux problèmes rencontrés. Il termine par une conclusion sur l'état du projet et des améliorations qui pourraient y être apportées.

\end{abstract}

\tableofcontents

\newpage

\section{Introduction}
\subsection{Préface}

Dans le cadre de notre préparation au diplôme d'ingénieur en Télécommunications et Réseaux, nous avons été amenés à effectuer notre projet de première année sous l'encadrement de Jérôme Ermont. Le but de ce projet était surtout de nous faire manipuler le langage C et la gestion de la mémoire mais il nous a également permis de modéliser une solution réseau simple.

La commutation de paquets, technique utilisée dans le transfert de données dans les réseaux informatiques, permet de rediriger les packets sur un lien physique particulier suivant des critères précis, avec éventuellement une modification de ceux-ci. Au vue de notre formation, le choix de la réalisation d'un simulateur de commutateur se trouve être pertinent et intéressant.

Ce rapport est composé de deux parties. La première partie présente l'application dans sa globalité, son fonctionnement. La seconde partie est consacrée au détail de l'implémentation. Enfin nous conclurons sur les apports personnels obtenus à la fin de ce projet.

\subsection{Rappel du sujet}

Nous avons effectué notre projet langage C de première année sous l'encadrement de Jérôme Ermont. Il était question principalement de simuler le fonctionnement d'un commutateur de niveau 3. 

Ainsi il s'agissait dans ce projet de simuler le travail d'un élément réseau dont le rôle serait de recevoir des trames, de les assembler pour reformer des paquets et d'effectuer par la suite la commutation de ces paquets. Les paquets routés sont à nouveau fragmentés et les trames obtenues sont stockées dans des queues selon leur priorité.

La première étape a été l'étude préalable de l'élément réseau impliqué, il s'agissait d'assimiler les fonctionnalités attendues de celui-ci et de, relativement à notre compréhension du sujet, réfléchir à un algorithme.

Dans un second temps, après nous être mis d'accord sur le modèle retenu nous nous sommes départagé le travail afin de gagner du temps.  

Nous avons ensuite regroupé notre travail et avons modifié la structure générale afin de l'optimiser au maximum. 

Enfin, il faut remarquer que notre application est vouée à évoluer : il serait par exemple souhaitable que le commutateur soit capable de mieux traiter les erreur (reprise sur erreur, renvoie...)

\section{Implémentation}
\subsection{Architecture globale de l'application}

Notre application ayant attrait au réseau, il est naturel de la décomposer en module indépendant à l'image des couches du modèle OSI. De plus, une telle décomposition rend notre solution évolutive. Il sera aisé de remplacer un module ou d'en ajouter. Sa décomposition en module assure aussi sa stabilité.

Le simulateur est lancé par la méthode \texttt{simulate} du module \texttt{simulator}. Les paramètres de lancement sont obtenue par le module \texttt{main}, en charge d'analyser les arguments de la ligne de commande. La simulation est commandé par le fichier d'entrée, comportant des instructions \texttt{IN} et des instruction \texttt{out}, dictant l'entrée ou la sortie d'un packet.

Lors de la lecture d'une instruction \texttt{IN}, le module \texttt{trame} sera appelé afin de lire une trame depuis le fichier d'entrée. Celle-ci est alors rangée dans une structure de données nommé \texttt{ListPacketFrag}, en attandant d'obtenir toute les trames d'un même paquet. C'est donc une liste chaîné, contenant dans chaque cellule un \texttt{PacketFrag} – paquet fragmenté – contenant un tableau de trame, remplit au fur et à mesure de leur arrivée. Lorsque toute les trames d'un même paquet sons reçu, le paquet fragmenté est retiré de la liste \texttt{ListPacketFrag}, afin d'être envoyé au module \texttt{packet}, en charge de l'assemblage. Le schéma ci-après résume le stockage des données dans les différentes structures.

\includegraphics{s1.png}

Un paquet étant reconstitué, celui-ci est envoyé au module \texttt{commut}, en charge de sa commutation. Celui-ci modifie son label en concordance avec la table de commutation, et indique au simulateur le lien de sortie. Celui-ci va alors demander au module \texttt{packet} de le fragmenter, avant de l'envoyer dans la \texttt{queue} dédié au lien de sortie précédement établit.

Lors d'une instruction \texttt{OUT}, le module \texttt{queue} est appelé afin d'envoyer une trame sur le lien de sortie concernée et donc de l'écrire dans un fichier de sortie. La trame sortie est sélectionnée selon sa priorité de manière circulaire sauf la priorité absolue, qui elle, est envoyée immédiatement après un \texttt{OUT}.

Le diagramme UML page suivante résume les différents modules utilisés dans l'application ainsi que les relations qui les lie.

\includegraphics[height=27cm]{UML.jpg}

\subsection{Détail des modules}

\subsubsection{trame}
Le module \texttt{trame} propose une fonction de décodage dont le prototype est le suivant :
\begin{minted}{c}
int trame_decode(Trame * t, FILE * stream);
\end{minted}
Son objectif est le décodage d'une trame depuis le fichier \texttt{stream} afin de remplir la structure suivante :
\begin{minted}[linenos]{c}
typedef struct {
    unsigned short int id; /* numéro du paquet */
    unsigned short int seqf; /* numéro de séquence */
    unsigned short int nbf; /* nombre de séquence */
    unsigned int length; /* longeur de charge */
    unsigned char * charge; /* contenue */
} Trame;
\end{minted}
Cette fonction a été implémenté sous la forme d'une machine à état afin
de suivre l'avancement dans la lecture. L'état $9$ désigne la lecture du
fanion de fin de trame, tandis qu'un état négatif désigne la détection
d'une erreur.
\begin{minted}[linenos]{c}
while (state < 9 && state > -1) {
    pending = fgetc(stream); /* Lecture d'un caractère */
    if (pending == EOF) { /* On contrôle qu'il n'y ai pas eu d'erreur */
        fprintf(stderr, "trame_decode: error in fgetc\n");
        state = -1; 
    } else {
        switch (state) {
            // Action suivant l'état
        }
    }
}
\end{minted}
Tout au long du décodage, la CRC est calculé en effectuant un ou exclusif
avec l'octet qui vient d'être lu. Si la CRC lu dans la trame n'est pas
identique, la trame est ignoré.
\begin{minted}{c}
crc ^= pending;
\end{minted}

%~ Le décodage se fait en telles étapes :
%~ \begin{enumerate}
 %~ \item initialiser une variable de crc afin d'effectuer le contrôle d'erreur
 %~ \item abandonner toutes les informations redondantes avant que l'on rencontre un fanion afin d'assurer le départ du décodage.
 %~ \item Pour chaque octet utile reçu, faire l'opération \textit{ou exclusif} avec l'ancien crc pour générer le nouveau crc.
 %~ \item Pour transformer les octets en attributs d'une trame, on décale le premier octet à gauche de 8 bits et l'opération \textit{ou} avec le deuxième octet afin de construire les attributs de 16 bits.
 %~ \item Lors que l'on rencontre un banaliseur, on récupère l'octet suivant.
 %~ \item Lors que l'on rencontre un fanion, la lecture de la séquence de bits est terminée et on récupère le dernier élément dans partie de charges comme le crc original.
 %~ \item Refaire l'opération \textit{ou exclusif} entre crc calculé et le crc original, afin d'obtenir un crc pour la trame sauf les fanion et le crc original.
 %~ \item Faire la comparaison entre le crc calculé et le crc original, s'ils sont identiques, la trame est reçue correctement, sinon, on ignore cette trame et elle est détruite.
%~ \end{enumerate}

Inversement, ce module propose une fonction d'encodage, servant cette
fois-ci à écrire les données contenue dans la structure \texttt{Trame}
dans un fichier. Voici son prototype :
\begin{minted}{c}
int trame_encode(FILE * stream, Trame * t);
\end{minted}


\subsubsection{packetfrag}
\texttt{packetfrag} correspond à un module gérant les fragments de paquets, ou les trames d'un même paquet. Un packetfrag contient, comme ci dessous, une trame ainsi que la taille du paquet auquel elle appartient :\begin{minted}{c}
struct PacketFrag {
    // Nombre de fragment (taille du tableau)
    int size;
    // Tableau contenant les trames
       Trame ** trames;
};
typedef struct PacketFrag PacketFrag;
\end{minted}


\subsubsection{listpacketfrag}
\texttt{listepacketfrag} est un module qui définit la structure des
maillons de la liste chainé qui contient les fragments de paquet.
\begin{minted}[linenos,tabsize=4]{c}
 struct ListPacketFrag {
    // Nombre de trame stocké dans le packet fragmenté
    int count;
    // ID du packet associé
    int id;
    // Packet fragmenté de la cellule
    PacketFrag * pf;
    // Cellule suivante
    struct ListPacketFrag * next;
};
typedef struct ListPacketFrag ListPacketFrag;
\end{minted}
\texttt{listepacketfrag} contient l'ID du paquet associé afin de savoir
si une trame appartient au paquet concerné ou non. Elle contient aussi
un compteur qui permet de savoir lorsque toute les trames d'un paquet
sont arrivées dès que celui-ci atteint le nombre de séquence.

Cette liste chaîné est utilisé afin de stoquer les paquets fragmentés
avant leur assemblage. Il propose donc des fonctions d'ajout de trame,
et de retrait de paquet fragmenté (lorsque le paquet est assemblé) dont
les prototypes sont les suivants :
\begin{minted}[linenos,tabsize=4]{c}
ListPacketFrag * lpf_add(ListPacketFrag ** lpf, Trame * t);
void lpf_del(ListPacketFrag ** lpf, ListPacketFrag * cellule);
\end{minted}


\subsubsection{packet}
\texttt{Packet} est un module contenant la structure qui représentera à nos yeux une trame définie comme ci dessous:
\begin{minted}[linenos,tabsize=4]{c}
 struct Packet {
  unsigned int label;
  unsigned char priority;
  unsigned int size;
  unsigned char * charge;
};
typedef struct Packet Packet;
\end{minted}
Ce modules est aussi chargé de l'assemblage et de la re-fragmentation du paquet en de multiple trames via les deux fonctions qui suivent :
\begin{minted}[linenos,tabsize=4]{c}
/* Assembler un packet */
void packet_assemble(Packet * packet, PacketFrag * pf);
/* Fragmenter un packet */
void packet_split(PacketFrag ** pf, Packet * packet, int outputsize);
\end{minted}




\subsubsection{commut}
\texttt{commut} est le module qui gère la table de commutation de notre appareil. Cette dernière est définie ainsi :
\begin{minted}{c}
 /* Liste chaîné des règles de commutation */
struct ListRule {
    unsigned int inlabel;
    unsigned int outlabel;
    int outlink;
    struct ListRule * next;
};
typedef struct ListRule ListRule;
\end{minted}
Pour charger la table de commutation, nous utiliserons :
\begin{minted}{c}
 /* Charger les règles de routage depuis un fichier */
int commut_load(ListRule ** rules, char * inputfile);
\end{minted}
Puis, la fonction :
\begin{minted}{c}
 /* Commuter un packet.
 * Change le label et renvoit le lien de sortie.
 * Renvoit -1 si le packet n'est concerné par aucune règle. */
int commut(Packet * packet, ListRule * rules);
\end{minted}
Qui elle, renvoie le lien de sortie et modifie le label du paquet.




\subsubsection{queue}
\texttt{Queue} est un module gérant les files d'attentes avec gestion
des priorité. Une queue est une liste chaîné de trames :
\begin{minted}{c}
 struct Queue {
    struct Queue * previous;
    Trame * element;
};
typedef struct Queue Queue;
\end{minted}
Chaque queue sera attribuée à une priorité. Dans notre cas, 256 priorités
sont géré, et chaque queue associé est stocké dans une cellule d'un
tableau de 256 queues, l'indice représentant la priorité.
Ce tableau est contenu dans une structure WRRS ci dessous :
\begin{minted}{c}
 /* Weighted Rount Robin Server */
struct WRRS {
    Queue * queue[NBPRIORITY]; // Tableau des queues de trame (une par priorité)
    int priority; // Priorité en cours
    int credit; // Nombre de crédit pour la priorité en cours
};
typedef struct WRRS WRRS;
\end{minted}
L'ajout de trame se fait par le biais de la fonction de la fonction
\texttt{wrrs\_trame}.
\begin{minted}{c}
/* Ajouter une trame dans le serveur */
void wrrs_push(WRRS * server, unsigned short int priority, Trame * trame);
\end{minted}
Les variables \texttt{priorité} et \texttt{credit} dans le WRRS
contiennent les informations nécessaire à l'application de l'algorithme
de décision lors de la sortie d'une trame. Celui-ci est déclanché par
l'appel de la fonction \texttt{wrrs\_pop} dont voici le prototype :
\begin{minted}{c}
/* Obtenir la prochaine trame devant sortir du serveur.
 * Retourne NULL si le server est vide. */
Trame * wrrs_pop(WRRS * server);
\end{minted}

\subsubsection{main}
Le \texttt{main} se contente de gérer les erreur et surtout de récupérer les arguments. Après avoir vérifié que les arguments sont correctes, ils les fait passer au module principal qui est \texttt{simulator}. Il vérifie aussi la validité des données.

\subsubsection{simulator}
\texttt{simulator} est le module qui appelle tous les autres modules précédemment énoncés. Il est appelé par le main et coordonne toutes les fonctions qui assureront le bon fonctionnement de notre commutateur de niveau 3. C'est lui qui ouvre et lit le fichier d'entrée et qui ouvre le fichier de sortie.


\section{Réalisation du projet}

\subsection{Difficultés et solutions}

Au départ, le stockage dans l'ordre des trames semblait bien difficile. Un tableau dynamique n'était pas simple à manipuler mais nous ne pouvions pas mettre de tableau fixe. Nous avons réalisé que lorsque l'on recevait une trame, on savait directement le nombre de trame que l'on recevrait grâce à l’attribut nbf. Ainsi il semblait bien plus simple d'allouer directement un tableau de la bonne taille et ranger les trames dans le bon ordre.

\subsubsection{Solutions envisageable}

Plusieurs solutions ont été proposées lors de l'étude du cahier des charges. Dans la mesure où il nous était demandé de ne pas nous focaliser sur les performances mais plutôt sur le choix des structures de données et l'efficacité des algorithmes.

Au départ nous comptions gérer les trames dans une partie décodage, gérer les paquets dans une partie assemblage et gérer la queue dans une partie éponyme.
 
Dans la partie décodage nous avions défini une structure de données Trame, une liste chaînée de trames utilisée pour stocker les trames d'un même paquet qui n'est pas encore complet et une liste de « Paquets Fragmentés » ListePacketFrag utilisée pour stocker les différentes listes de trame.


Nous avions alors envisagé que les opérations disponibles sur une trame pourraient être :
Créer une trame à partir d'un numéro de paquet, d'un numéro de séquence, d'un nombre de séquence, de la longueur d'une charge, d'une somme de contrôle et d'un contenu.
\begin{enumerate}
 \item Initialiser une trame
 \item Transformer une séquence de bits en trame
 \item Transformer une trame en une séquence de bits
 \item Vérifier si un ID existe
 \item Ajouter une trame dans la ListeTrame
 \item Ajouter une ListeTrame dans la ListePacketFrag
 \item Vérifier si la ListeTrame est complète
 \item Vérifier si la Trame est correcte en utilisant CRC
 \item Calculer CRC
\end{enumerate}

C'est dans la partie assemblage que nous avions défini la structure Packet et les opérations disponibles sur un paquet, la plus importante étant bien sûr le décodage de la trame dans le but de l'intégrer à un paquet.

Puis dans la partie queue nous avions envisagé que la queue serait un tableau de taille 256, soit le nombre maximal de priorité. 

% TODO

\subsubsection{Difficultés rencontrées et solution retenue}

Comme le calcul de CRC ne tient pas compte du CRC original suivant la charge, donc les difficultés concernant cette procédure sont :
\begin{enumerate}
 \item Comment faire un calcul de CRC en excluant le CRC original.
 \item Comment faire un calcul de CRC en cas ayant une CRC égale à un fanion ou un banaliseur.
\end{enumerate}


\vspace{0.5cm}

Pour le problème 1 :

selon la formule :

A = A xor B xor B

on sait qu'il suffit de faire une fois ou exclusif avec le CRC original et le CRC calculé incluant CRC original, on peut obtenir un CRC  en excluant le CRC original.

Pour le problème 2 :

Si la CRC reçu est un fanion ou un banaliseur, c'est qu'on a
 calculé la crc aussi sur le banaliseur qui précédait la crc reçu.

Donc on doit faire encore une fois ou exclusif avec le banaliseur.


Comme il n'y pas de mécanisme de gestion d'exception comme JAVA pour langage C, on utilise un type de retour INT pour indiquer l'état d'exécution afin de contrôler les exceptions.
On a décider un code de retour tel que si 0, ça veut dire qu'il n'y pas d'erreur, -1 sinon.

\vspace{0.5cm}

Ainsi, partant de la volonté de développer une application efficace, nous avons effectué grand nombre de modifications. La plus significative concernant la structure de donnée qui stocke les trames lorsqu'un paquet n'est pas encore complet. En effet, nous sommes passés d'une liste de liste à une liste de tableau.

Un second regard nous a permis de réaliser que lorsque l'on recevait une trame, on savait directement le nombre de trame que l'on recevrait grâce à l'attribut nbf. Ainsi il semblait bien plus simple d'allouer directement un tableau de la bonne taille et ranger les trames dans le bon ordre.

\section{Conclusion}

Pour mener à bien ce projet de première année, nous avons dû approfondir nos connaissances en terme d'analyse de trame. Nous nous sommes aussi documenté sur le fonctionnement des protocoles TCP/IP afin de s'en inspirer.

De plus, ce projet nous a permis de nous familiariser avec la démarche de création d’un programme complexe ainsi  que des notions vues en réseaux. En effet, nous avons développé cette application par un travail de groupe, ce qui nous attend en tant que futurs ingénieurs. Plusieurs personnes travaillant sur un même programme ont besoin d'énormément de coordination et de synchronisation. Cela s'est révélé bien plus difficile que prévu, outre la standardisation de nos variables et la répartition des modules.

La partie que nous avons développée correspond aux objectifs de départ. Les résultats de simulation sont tout à fait satisfaisants tant au niveau routage qu'au niveau gestion de la mémoire. Savoir faire des simulation et le faire tout le long du projet est un outil précieux afin de corriger les erreurs le plus vite possible avant que celles-ci ne se répercutent sur la suite.

Le projet que nous avons réalisé cette année peut encore évoluer.
En effet, la gestion des trames perdues ou erronées est très limitée.
Il pourrait être envisagé d'étendre le protocole afin de permettre de la
reprise sur erreur. Par exemple, nous pourrions une fonctionnalité qui détruirait la trame si aucune autre trame du même paquet 
n'arrive dans un délai imparti. Ceci permet de recevoir un paquet de
même ID qui arriverait plus tard. Une autre voie d'évolution est la
gestion de plusieurs liens entrants, avec éventuellement une table de
commutation qui en tiendrait compte. Il serait également intéressant de
passer à une version asynchrone permettant la réception et l'émission de
trame simultanément.

\end{document}
